\chapter[PENDAHULUAN]{\\ PENDAHULUAN}

\section{Latar Belakang}
Website mengalami perkembangan yang signifikan. Bukan hanya sebagai media untuk menampilkan informasi berupa teks atau gambar sederhana, saat ini, website telah menjadi platform dengan ekosistem yang kompleks dan diperuntukkan untuk berbagai hal. Website dapat dikembangkan dengan beberapa framework dan salah satu framework yang sering digunakan adalah Laravel. Laravel merupakan kerangka kerja PHP yang dapat diimplementasikan dalam berbagai proyek [1], salah satunya adalah sebagai dashboard yang menampilkan informasi dari perangkat lain seperti contohnya sistem IoT. Dalam sistem IoT, website dashboard memegang peranan sebagai antarmuka visual yang menampilkan visualisasi data penting dari perangkat IoT.

Dalam konteks ini, visualisasi data menjadi media yang sering digunakan untuk memahami dan mengambil insight dari data-data perangkat IoT yang memiliki ukuran besar serta memperoleh kesimpulan dalam pemrosesannya baik dalam bentuk plot, chart, atau bahkan animasi [2], [3]. Selain itu, melalui visualisasi data yang interaktif dan informatif, pengguna dapat dengan cepat mendeteksi masalah, menganalisis performa, serta melakukan penyesuaian atau pengaturan terhadap perangkat IoT tanpa perlu terlibat dalam proses teknis yang rumit. 

Untuk membuat visualisasi data dari perangkat sistem IoT, terdapat banyak opsi library yang dapat diimplementasikan dengan beberapa bahasa pemrograman. Seperti contohnya library Plotly yang dapat diimplementasikan menggunakan bahasa pemrograman Python [3] atau library visualisasi seperti D3 dan Chart.js yang diimplementasikan dengan bahasa pemrograman JavaScript [4]. Terdapat beberapa library JavaScript yang dapat digunakan untuk memvisualisasikan data sensor IoT, beberapa diantaranya adalah Chart.js, D3.js, dan Highcharts yang ketiganya memiliki spesifikasi, kelebihan, dan kekurangan masing-masing. Chart.js adalah library visualisasi data berbasis JavaScript yang sederhana dan open-source. Library ini cocok untuk proyek kecil dan menengah dan mendukung berbagai jenis grafik seperti bar, line, dan pie [5]. D3 (atau D3.js) adalah library JavaScript gratis dan open-source yang digunakan untuk membuat visualisasi data. Dengan fitur-fiturnya, D3 mendukung fleksibilitas dalam membangun grafik yang interaktif dan dinamis. D3 menawarkan penggunaan stylesheet eksternal untuk mengubah tampilan grafik termasuk menyesuaikan dengan media queries untuk grafik responsif atau mode gelap [6]. Sedangkan Highcharts adalah library chart yang banyak digunakan untuk memvisualisasikan data berskala besar dan kompleks serta mendukung grafik fitur yang memiliki performa tinggi [7]. Berdasarkan data yang diambil dari website ossinsight.io saat data ini diakses, Chart.Js, D3.js, dan Highcharts berada pada 10 besar library yang paling populer berdasarkan jumlah stars dan pull request pada github [8].

Oleh karena itu, untuk menentukan library yang tepat dalam proses visualisasi pada suatu website, terdapat beberapa hal yang menjadi pertimbangan. Misalnya, bahasa pemrograman yang digunakan, grafik interaktif atau statis, dengan animasi atau tanpa animasi, dan performa rendering. Dengan pertimbangan tersebut, JavaScript menjadi media paling fleksibel untuk menampilkan grafik statis maupun interaktif karena JavaScript telah diimplementasikan di setiap browser modern dan tidak memerlukan dependensi tambahan dalam penggunaannya. Selain itu, ketika digabungkan dengan CSS, HTML5, dan library yang tepat, kemampuan JavaScript menjadi semakin luas dan berkembang [4], [9].

Penelitian ini  akan membandingkan tiga library JavaScript yaitu Chart.js, D3.js, dan Highcharts karena ketiga library tersebut termasuk kedalam 10 besar library charting JavaScript dan memiliki jumlah pull request di atas 1000 [8]. Selain karena popularitasnya yang sering digunakan oleh pengembang, alasan untuk membandingkan ketiga library ini adalah guna membantu pengembang memilih library visualisasi data yang paling sesuai dengan kebutuhannya untuk menangani data IoT yang secara terus menerus menghasilkan data baru dengan kecepatan yang tinggi [10]. Pemilihan library dievaluasi berdasar beberapa aspek utama, seperti performa rendering data real-time, skalabilitas chart, serta efisiensi dalam hal penggunaan sumber daya CPU dan memori. Evaluasi tersebut diukur berdasarkan efisiensi sistem dalam mengeksekusi suatu tugas, termasuk salah satunya adalah pengukuran performa real-time data rendering untuk mengukur waktu yang dibutuhkan oleh website dalam menampilkan visualisasi [4] serta analisis penggunaan memori dan CPU saat web diakses.

Pengujian dilakukan dengan studi kasus pengembangan dashboard untuk perangkat IoT. Website ini dibuat untuk mempermudah pengguna perangkat IoT dalam menganalisis data-data dari sensor IoT dalam bentuk visualisasi grafik yang mudah dipahami. Dengan adanya penelitian ini, diharapkan pengembang dapat memilih library yang paling optimal untuk visualisasi data, dan memastikan performa dashboard yang lebih baik dari segi user experience maupun efisiensi sistem.

\section{Rumusan Masalah}
Berdasarkan latar belakang yang telah diuraikan sebelumnya, maka dirumuskan beberapa rumusan masalah dalam karya tulis ini, sebagai berikut :
\begin{enumerate}
    \item Bagaimana performa rendering website saat data IoT divisualisasikan dengan library Highcharts, Chart.js, atau D3? 
    \item Bagaimana penggunaan memori dan CPU saat data dimuat dan divisualisasikan dengan library Highcharts, Chart.js, atau D3? 
    \item Bagaimana skalabilitas chart yang dibuat dengan library Highcharts, Chart.js, atau D3.js? 
    \item Library manakah yang memiliki performa paling baik untuk memvisualisasikan data IoT pada kasus ini? 
\end{enumerate}

\section{Batasan Masalah}
Agar penelitian tetap terfokus pada ruang lingkup yang telah ditetapkan \cite{muhammadred}, akan dijelaskan batasan-batasan yang perlu dipertimbangkan dalam pengembangan penelitian ini sebagai berikut:
\begin{enumerate}
    \item Data yang digunakan pada percobaan ini merupakan dataset numerik dan tidak mencakup data IoT yang berupa gambar atau audio. 
    \item Penelitian ini dilakukan menggunakan perangkat keras dan perangkat lunak yang terbatas, dengan spesifikasi komputer tertentu yang digunakan untuk mengukur performa. Hasil penelitian ini mungkin tidak sepenuhnya representatif untuk perangkat keras atau perangkat lunak yang memiliki spesifikasi yang berbeda.
    \item Library yang digunakan dalam penelitian ini hanya berfokus pada Highcharts, Chart.js, atau D3 serta hanya meneliti penggunaan line chart
    \item Data yang digunakan sebagai bahan perbandingan pada penelitian ini merupakan data dummy yang dibuat dengan format seperti data sensor sebenarnya.
    \item Pengujian ini menggunakan 5000 data uji untuk meninjau indikator-indikator yang telah ditentukan. 
\end{enumerate}

\section{Tujuan Penelitian}
Berikut adalah beberapa tujuan penelitian yang telah ditetapkan untuk memandu jalannya penelitian ini: 
\begin{enumerate}
    \item Mengetahui library yang paling efisien untuk memvisualisasikan data IoT diantara library Highcharts, Chart.js, atau D3.js dengan pertimbangan parameter yang telah ditentukan.
    \item Memperoleh cara terbaik untuk memaksimalkan penggunaan library tersebut.
\end{enumerate}

\section{Manfaat Penelitian}
Adapun manfaat-manfaat yang diperoleh dari penelitian ini, yaitu:
\begin{enumerate}
    \item Mempermudah pengembang perangkat lunak untuk mengembangkan dasbor IoT yang menampilkan visualisasi berupa chart. 
    \item Penelitian ini dapat memberikan wawasan tentang bagaimana library visualisasi mempengaruhi kinerja runtime, yang dapat digunakan untuk meningkatkan efisiensi dalam memvisualisasikan data.
    \item Membantu meningkatkan pengalaman pengguna dalam interaksi dengan dashboard dan sistem IoT.
\end{enumerate}

\section{Sistematika Penulisan}
Laporan proyek akhir ditulis dengan sistematika penulisan sebagai berikut.
\begin{enumerate}
    \item Bab I Pendahuluan: menjelaskan latar belakang, rumusan masalah, tujuan, batasan masalah, serta sistematika penulisan dari laporan proyek akhir.
    \item Bab II Tinjauan Pustaka:Bab ini menjelaskan referensi pendukung dari penelitian yang dilakukan dan perbandingan dengan penelitian-penelitian sejenis yang membahas mengenai perbandingan library JavaScript untuk merepresentasikan data-data IoT. 
    \item Bab III Metodologi Penelitian: Menjelaskan tentang alat dan bahan dalam penelitian, serta matrik evaluasi yang digunakan dalam membandingkan ketiga library JavaScript yang dimaksud.  
    \item Bab IV Hasil dan Pembahasan: Menjelaskan hasil pengujian dari library Chart.js, D3, dan Highcharts berdasarkan parameter-parameter yang digunakan.
    \item Bab V Kesimpulan dan Saran: Menjelaskan hasil pengujian dari library Chart.js, D3, dan Highcharts berdasarkan parameter-parameter yang digunakan.
    \item Daftar Pustaka: menjabarkan sumber-sumber yang digunakan dalam laporan proyek akhir.
\end{enumerate}

Secara keseluruhan, sistematika penulisan dalam laporan proyek akhir sarjana terapan adalah susunan atau struktur dari laporan proyek akhir sarjana terapan yang menjabarkan bagian-bagian yang harus ada dalam laporan proyek akhir sarjana terapan, yang meliputi Pendahuluan, Tinjauan Pustaka, Metode Penelitian, Hasil dan Pembahasan, Kesimpulan dan Saran, serta Daftar Pustaka. Sistematika penulisan yang baik akan membuat laporan proyek akhir sarjana terapan lebih mudah untuk dibaca dan dipahami.