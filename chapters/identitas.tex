%Isi identitas proyek akhir disini

\providecommand{\judulid}{ANALISA PERBANDINGAN KINERJA LIBRARY CHART.JS, D3 DAN HIGHCHARTS UNTUK VISUALISASI DATA SENSOR IOT PADA DASHBOARD BERBASIS LARAVEL } %Judul Tugas Akhir Bahasa Indonesia
\providecommand{\judulen}{Judul Proyek Akhir (English)} %Judul Tugas Akhir Bahasa Inggirs
\providecommand{\penulis}{Ailsa Isnani Anubhawa} %Nama Lengkap Mahasiswa 
\providecommand{\nim}{21/474745/SV/19024} % NIM

\providecommand{\tipe}{Proyek Akhir} %Tipe Laporan
\providecommand{\type}{Final Project} %Tipe Laporan
\providecommand{\prodi}{Teknologi Rekayasa Perangkat Lunak} %Nama Prodi
\providecommand{\departemen}{Departemen Teknik Elektro dan Informatika}
\providecommand{\fakultas}{Sekolah Vokasi} %Nama Fakultas
\providecommand{\universitas}{Universitas Gadjah Mada} %Nama Universitas
\providecommand{\tglpengesahan}{\today} %Tanggal di Halaman Pengesahan
\providecommand{\tglpersetujuan}{\today} %Tanggal di Lembar Persetujuan
\providecommand{\tglpernyataan}{\today} %Tanggal di Surat Pernyataan

\providecommand{\tahun}{\the\year{}} %Tahun Proyek Akhir

\providecommand{\ketuapenguji}{Nama Ketua Penguji} %Nama Ketua Penguji
\providecommand{\NIPketuapenguji}{XXXXXXXXXXXXXXXXXX} %NIP Ketua Penguji

\providecommand{\sekretarispenguji}{Nama Sekretaris Penguji} %Nama Sekretaris Penguji
\providecommand{\NIPsekretarispenguji}{XXXXXXXXXXXXXXXXXX} %NIP Sekretaris Penguji

\providecommand{\anggotapenguji}{Nama Anggota Penguji} %Nama Anggota Penguji
\providecommand{\NIPanggotapenguji}{XXXXXXXXXXXXXXXXXX} %NIP Anggota Penguji

\providecommand{\koordepartemen}{Nama Ketua Departemen} %Nama Koordepartemen
\providecommand{\NIPkadep}{XXXXXXXXXXXXXXXXXX} %NIKA Koordepartemen

\providecommand{\koorprodi}{Nama Ketua Prodi} %Nama Koorprodi
\providecommand{\NIPKaprodi}{XXXXXXXXXXXXXXXXXX} %NIP Koorprodi

\providecommand{\pembimbing}{Nama Dosen Pembimbing} %Nama Dosen Pembimbing
\providecommand{\NIPpembimbing}{XXXXXXXXXXXXXXXXXX} %NIP Dosen Pembimbing

\providecommand{\katakunci}{kata kunci 1, kata kunci 2, kata kunci 2} %Kata kunci dalam Bahasa Indonesia
\providecommand{\keywords}{keyword 1, keyword 2, keyword 3} %Kata kunci dalam Bahasa Inggris
