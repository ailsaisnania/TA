%\chapter{ABSTRAK}
\clearpage
\phantomsection
\addcontentsline{toc}{chapter}{INTISARI}
\begin{center}
   % \textbf{\large{\judulid}}\\[0.5cm]
   % Oleh :\\
   % \penulis\\
   % \nim\\[2em]
    \textbf{INTISARI}\\[0.5cm]
\end{center}

Website telah berkembang menjadi platform yang kompleks, salah satunya yaitu dengan penggunaanya sebagai dashboard untuk menampilkan informasi dari perangkat IoT. Dalam hal ini, website dashboard berfungsi sebagai antarmuka visual untuk menampilkan data penting IoT dan memvisualisasikan data-data tersebut. Untuk membuat visualisasi data IoT, terdapat beberapa library yang dapat digunakan dan setiap library memiliki kelebihan serta kekurangannya masing-masing. Dengan beberapa pertimbangan seperti performa rendering, fleksibilitas, dan implementasi, library dari bahasa pemrograman JavaScript dapat menjadi salah satu pilihan karena fleksibilitasnya yang dapat dijalankan di semua browser modern tanpa membutuhkan dependensi tambahan. Penelitian ini membandingkan tiga library JavaScript —Chart.js, D3.js, dan Highcharts— untuk membantu pengembang memilih yang paling sesuai dalam menangani data IoT yang terus-menerus diperbarui. Evaluasi penggunaan library visualisasi JavaScript diukur berdasarkan performa rendering data real-time, skalabilitas tampilan chart, dan efisiensi penggunaan CPU serta memori. Hasil penelitian ini diharapkan dapat memberikan pengetahuan dan panduan dalam memilih library visualisasi yang optimal untuk meningkatkan performa dashboard dan efisiensi sistem

\noindent Kata kunci: \katakunci